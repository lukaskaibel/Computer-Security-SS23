The paper discusses cache attacks as a type of side-channel attacks that leverage cross-process information leaking through CPU caches.

The attacks are possible due to how \textit{set-associative memory caches} are implemented.
Their structure limits the mapping of memory addresses to the cache in the sense a) always full memory blocks are cached, and b) that a memory block can only be cached in specific cache lines.

The attacked cipher is the AES cipher, where the attacks leverages how it is implemented in performance-oriented software.
Concrete, the implementation pre-computes Lookup-tables. These tables are then used to realize the algebraic Shift-Rows, Mix-Columns and Sub-Bytes operations needed in each round of the AES computation.

The paper describes multiple attacks where a process can obtain information about another process' memory access patterns by leveraging said implementation details of the the memory caches and AES implementation.
Finally, the paper also discusses countermeasures for such attacks.




